\documentclass[a4paper,12pt]{article}
\usepackage[spanish]{babel}
\hyphenation{co-rres-pon-dien-te}
\usepackage[utf8]{inputenc}
\usepackage[T1]{fontenc}
\usepackage{graphicx}
\usepackage[pdftex,colorlinks=true, pdfstartview=FitH, linkcolor=blue,
citecolor=blue, urlcolor=blue, pdfpagemode=UseOutlines, pdfauthor={H. Asorey},
pdftitle={F3B+F4B - Guía 03}]{hyperref}
\usepackage[adobe-utopia]{mathdesign}

\hoffset -1.23cm
\textwidth 16.5cm
\voffset -3.5cm
\textheight 26.0cm

%----------------------------------------------------------------
\begin{document}
\title{
{\normalsize{Universidad Nacional de Río Negro - Profesorado de Física}}\\
Física 3B+4A  2018 \\ Guía 06: Aplicaciones 
}
\author{Asorey}
\date{14 de Junio de 2018} 
\maketitle

\begin{enumerate}
	\setcounter{enumi}{48}      %% esta guia va del problema 39 al XX
	\item{\bf{Tibio, tibio...}}

		Considere una pared de espesor $d=0.15$\,m hecha con un vidrio especial
		que tiene las siguientes propiedades: Conductividad térmica
		$k=0.78$\,W\,m$^{-1}$\,K$^{-1}$; Densidad $\rho = 2700$\,kg\,m$^{-3}$;
		Calor específico $C_p = 0.84$\,kJ\,kg$^{-1}$\,K$^{-1}$.  Las caras de
		esa pared se mantienen a $T_i=29.5^\mathrm{o}$C y
		$T_i=18.0^\mathrm{o}$C respectivamente. Determine el flujo de calor por
		metro cuadrado a través de la pared.
	
	\item{\bf{Patinando en el cerro}}
		
		Durante el invierno, en la superficie de la laguna Frey se forma una
		capa de hielo ($k=2$\,W\,m$^{-1}$\,K$^{-1}$) de $d=0.25$\,m de espesor.
		Sabiendo que: la temperatura media del aire sobre el hielo es de
		$T_\mathrm{ext}=272$\,K; la temperatura media del agua bajo el hielo es
		de $T_\mathrm{int}=277$\,K; La superficie de la laguna es de
		$S=100$\,m$^2$ y la profundidad media es de $h=10$\,m; y es posible
		despreciar la radiación solar. Calcule:
		\begin{enumerate}
			\item la cantidad total de calor almacenada en el agua líquida;
			\item la cantidad de calor por segundo que irradia la laguna al
				aire circundante;
			\item el tiempo necesario para que la temperatura del agua líquida
				descienda a $2$\,$^\mathrm{o}$C.
			\item el tiempo necesario para que la laguna se congele.
			\item el cambio de entropía del universo debido a este proceso.
			\end{enumerate}
		
	\item{\bf{Refrigeración}}
		
		Una casa fabricada con paredes de mampostería
		($k=0.8$\,W\,m$^{-1}$\,K$^{-1}$, $d=0.20$\,m) tiene una superficie
		total de paredes de $200$\,m$^2$. Suponemos que en esta casa todas las
		pérdidas importantes se dan a través de las mismas, considerando al
		techo, a los cimientos y a las aberturas como aislantes perfectos
		($k=0$).  En verano, la temperatura exterior es
		$T_{\mathrm{amb}}=308$\,K y se pretende que la interior sea
		$T_{i}=293$\,K. El arquitecto dispone de varios equipos de aire
		acondicionado de $2500$\,W.
		\begin{enumerate}
			\item Calcule qué cantidad de calor por segundo ingresa a la casa.
			\item Determine el número de equipos de aire necesarios para lograr
				el objetivo.
			\item Calcule la temperatura de equilibrio una vez que los equipos
				estén instalados.
		\end{enumerate}
	
	\item{\bf{Aislantes}}
		
		En Bariloche se quiere construir una casa cuyas paredes cubren un área
		total de $150$\,m$^2$ y están hechas de un material multicapa. El mismo
		consiste en (de afuera hacia adentro):
		\begin{enumerate}
			\item Placa cementicia, $k=0.8$; $d=0.008$\,m;
			\item Placa de madera, $k=0.5$; $d=0.015$\,m;
			\item Aire, $k=0.02$; $d=0.02$\,m;
			\item Lana de vidrio, $k=0.04$; $d=0.07$\,m;
			\item Placa de yeso, $k=0.7$; $d=0.013$\,m.
		\end{enumerate}
		Suponemos que en esta casa todas las pérdidas importantes se dan a
		través de las paredes, considerando al techo, a los cimientos y a las
		aberturas como aislantes perfectos ($k=0$). En invierno, la temperatura
		exterior es $T_{\mathrm{amb}}=270$\,K, mientras que en verano es
		$T_{\mathrm{amb,v}}=310$\,K, y se desea que en invierno la temperatura
		interior sea $T_i=293$\,K.
		\begin{enumerate}
			\item Determine la resistencia equivalente $\rho$ de las paredes de
				la casa.
			\item Calcule la pérdida de calor que sufre la casa en invierno.
			\item Determine el número de estufas de $3000$\,kcal/hora que
				deberán instalarse para mantener la temperatura interior
				deseada durante el invierno ($1$\,kcal\,hora$^{-1}$ =
				$1.16$\,W).
			\item Calcule la temperatura de equilibrio una vez que las estufas
				funcionan.
			\item Si en verano, al mediodía la temperatura en el interior es
				$T_i=295$\,K y en el exterior es $T_{\mathrm{amb}}=310$\,K,
				calcule el tiempo necesario para que la temperatura interior
				alcance la temperatura exterior.
		\end{enumerate}
	
	\item {\bf{Uy, ¡que frío!}}
		
		Una cámara frigorífica debe mantener una temperatura de
		$T_i=-25^\mathrm{o}$\,C con una temperatura exterior de
		$T_2=30^\mathrm{o}$\,C. La pared de la cámara se construye de la
		siguiente manera:
		\begin{itemize}
			\item Revoque de $2$\,cm de espesor
				($k=0.9$\,W\,m$^{-1}$\,K$^{-1}$).
			\item Ladrillo macizo de $25$\,cm ($k=0.7$\,W\,m$^{-1}$\,K$^{-1}$).
			\item Telgopor de $x$\,cm ($k=0.06$\,W\,m$^{-1}$\,K$^{-1}$).
			\item Revoque de $2$\,cm de espesor
				($k=0.9$\,W\,m$^{-1}$\,K$^{-1}$).
		\end{itemize}
		Si la pérdida de calor no debe superar las $12$\,W\,m$^{-2}$, se pide
		determinar:
		\begin{enumerate}
			\item El coeficiente global de transmisión de calor que debe tener
				la pared.
			\item El espesor de Telgopor que debe colocarse.
		\end{enumerate}

	\item {\bf{Resistencias}}

		Demuestre las expresiones obtenidas en clase para el caso de las
		resistencias térmicas en paralelo. Luego proponga un ejemplo práctico
		de uso de las mismas. 
	
	\item {\bf{Aislaciones}}
		
		Se debe calefaccionar una casa cuya superficie total expuesta es
		$A=200$\,m$^2$, y se pretende limitar el consumo de gas a
		$G=260$\,m$^3$ mensuales, manteniendo la casa a $T_i=293$\,K, aún en
		julio cuando la temperatura exterior es $T_e=263$\,K. La calefacción
		funcionará en forma continua las 24 horas.
		
		Las paredes de la casa son de mampostería revocada
		($k=0.9$\,W\,m$^{-1}$\,K$^{-1}$) de $0.2$\,m de espesor, y se colocará
		una capa interna de lana de vidrio ($k=0.08$\,W\,m$^{-1}$\,K$^{-1}$) de
		$x$\,m de espesor, revestida con placas de Durlok
		($k=0.3$\,W\,m$^{-1}$\,K$^{-1}$) de $0.01$\,m de espesor.
		
		Determine:
		\begin{enumerate}
			\item la potencia disipada máxima admisible para esta casa
			\item el coeficiente global de transmisión de calor que debe tener
				la pared.
			\item el espesor $x$ de lana de vidrio que debe colocarse.
		\end{enumerate}
		Recuerde: el poder calorífico del gas es $4\times 10^7$\,J; los
		segundos en un mes son:$24\times3600\times30$\,s$=2592000$\,segundos
		por mes.

\end{enumerate}
\end{document}
%%%%
