\documentclass[a4paper,12pt]{article}
\usepackage[spanish]{babel}
\hyphenation{co-rres-pon-dien-te}
\usepackage[utf8]{inputenc}
\usepackage[T1]{fontenc}
\usepackage{graphicx}
\usepackage[pdftex,colorlinks=true, pdfstartview=FitH, linkcolor=blue,
citecolor=blue, urlcolor=blue, pdfpagemode=UseOutlines, pdfauthor={H. Asorey},
pdftitle={F3B+F4B - Guía 03}]{hyperref}
\usepackage[adobe-utopia]{mathdesign}

\hoffset -1.23cm
\textwidth 16.5cm
\voffset -3.5cm
\textheight 26.0cm

%----------------------------------------------------------------
\begin{document}
\title{
{\normalsize{Universidad Nacional de Río Negro - Profesorado de Física}}\\
Física 3B+4A  2018 \\ Guía 05: Segundo principio
}
\author{Asorey}
\date{30 de Mayo de 2018} 
\maketitle

\begin{enumerate}
	\setcounter{enumi}{38}      %% esta guia va del problema 39 al XX
    \item {\bf{El cuadrado inverso}}
		Una máquina frigorífica está equipada con $n=1000$\,moles de un gas
		ideal di-atómico inicialmente ocupando un volumen $V_A=112$\,m$^3$ a una
		temperatura $T_A=6825$\,K, y que opera con el siguiente ciclo: 1)
		compresión isobárica hasta llegar a un quinto del volumen inicial; 2) enfriamiento isocórico hasta alcanzar un quinto de la temperatura inicial; 3) expansión isobárica; 4) calentamiento isocórico.
        \begin{enumerate}
            \item Complete el cuadro de estados, encontrando los valores de
                $P$, $V$, $T$ y $n$ para cada uno de los estados $A$, $B$, $C$
                y $D$.
            \item En el diagrama $PV$ ubique los estados y dibuje las
				transformaciones experimentadas por el gas. Luego hágalo para
				un diagrama $TS$.
            \item Complete el cuadro de transformaciones, encontrando los
                cambios de energía interna, calor, trabajo y entropía en cada
                transformación.
			\item Calcule el rendimiento de la máquina frigorífica usando la
				definición de una máquina térmica, pero adaptada a este caso:
				$\eta=W_{\mathrm{neto}}/Q_{<0}$.
			\item Compare los valores obtenidos con los correspondientes del
				ejercicio 27 (Guía 03).
			\item Calcule el cambio de entropía total del Universo. Explique.
        \end{enumerate}
        {\bf{R}}: a) 
		$P_A=506634,4$\,Pa, $V_A=112$\,m$^3$, $n_A=1000$\,mol, $T_A=6825$\,K; 
		$P_B=506634,4$\,Pa, $V_B=22,4$\,m$^3$, $n_B=1000$\,mol, $T_B=1365$\,K; 
		$P_C=101325$\,Pa, $V_C=22,4$\,m$^3$, $n_C=1000$\,mol, $T_C=273$\,K; 
		$P_D=101325$\,Pa, $V_D=112$\,m$^3$, $n_D=1000$\,mol, $T_D=1365$\,K. 
		c) 
		$Q_1=-158,9$\,MJ, $\Delta U_1=-113,5$\,MJ, $W_1=-45,4$\,MJ;
		$Q_2=-22,7$\,MJ, $\Delta U_2=-22,7$\,MJ, $W_2=0$;
		$Q_3=31,8$\,MJ, $\Delta U_3=22,7$\,MJ, $W_3=9,1$\,MJ; 
        $Q_4=113,5$\,MJ, $\Delta U_4=113,5$\,MJ, $W_4=0$. 
		d) $\eta=(-36,3)/(-181,6) = 0,2 = 20\%$; 
		f) $\Delta S_{\mathrm{sis}} = 0$; $\Delta
		S_{\mathrm{med}}=159,64$\,kJ/K; $\Delta S_U=159,64$\,kJ/K $>0 \to$
		irreversible. Ayuda: $\Delta S_{\mathrm{med}} = +\frac{|Q_1|}{T_B} +
		\frac{|Q_2|}{T_C} - \frac{|Q_3|}{T_D} - \frac{|Q_4|}{T_A}$.
	
	\item {\bf{Entropía en aumento, 1}}

		Para mantener la temperatura dentro de una casa a $293$\,K se necesita
		mantener funcionando un sistema de calefacción con una potencia térmica
		de $30$\,kW por día cuando la temperatura en el exterior es de
		$266$\,K.  ¿Cuál es la tasa de incremento de la entropía total del
		Universo provocada por esta casa?
		\\{\bf{R}}: $\Delta S_U=+10,4$\,W/K $\to$ irreversible.

	\item {\bf{Variación de entropía, I}}

		¿Cuál es la variación de entropía de $1$\,mol de un gas ideal
		monoatómico si su temperatura aumenta de $100$\,K a $300$\,K en una
		transformación: a) isocórica; b) isobárica; c) adiabática y reversible.
		\\{\bf{R}}: a) $\Delta S= 13,7$\,J/K; b) $\Delta S=22,9$\,J/K; c)
		$\Delta S=0$.
	
	\item {\bf{Variación de entropía, II}}
		
		Calcule la variación de entropía cuando un mol de un gas ideal
		monoatómico se lleva desde $T_A=273$\,K y $P_A=2$\,atm hasta un estado
		$T_B=233$\,K y $P_B=0,4$\,atm.
		\\{\bf{R}}: $\Delta S= 10,1$\,J/K

	\item {\bf{Ciclo ireversible}}

		Una determinada cantidad de Helio está inicialmente a una presión
		$P_A=16$\,atm, $T_A=600$\,K y ocupa un volumen $V_A=1$\,L. Se somete a
		una expansión isotérmica de manera cuasiestática (reversible) hasta un
		volumen $V_B=4,0$\,L. Luego, es comprimido cuasi-estática e
		isobáricamente hasta que su volumen $V_C$ y temperatura $T_C$ le
		permiten volver de manera adiabática y reversible al estado original
		$A$. 
		\begin{enumerate}
			\item Complete el cuadro de estados y de transformaciones. Incluya
				para este último los cambios de entropía en cada
				transformación. 
			\item Realice un esquema del ciclo en un diagrama $PV$ y en un
				diagrama $TS$. 
			\item Calcule la eficiencia del ciclo; 
			\item Calcule el cambio de entropía del Universo
		\end{enumerate}
        {\bf{R}}: a) 
		$P_A=1621200$\,Pa, $V_A=0,001$\,m$^3$, $n_A=0,325$\,mol, $T_A=600$\,K; 
		$P_B=405307,5$\,Pa, $V_B=0,004$\,m$^3$, $n_B=0,325$\,mol, $T_B=600$\,K; 
		$P_C=405307,5$\,Pa, $V_C=0,0023$\,m$^3$, $n_C=0,325$\,mol, $T_C=344,6$\,K;
		$Q_1=2247,5$\,J, $\Delta U_1=0$, $W_1=2247,5$\,J, $\Delta S_1=3,75$\,J/K
		$Q_2=-1725,2$\,J, $\Delta U_2=-1035,1$\,J, $W_2=-690,1$\,J, $\Delta S_2=-3,75$\,J/K
		$Q_3=0$, $\Delta U_3=1035,1$\,J, $W_3=-1035,1$\,J, $\Delta S_3=0$.
		c) $\eta=(522,3$\,J$/2247,5$\,J$ = 0,232 = 23,2\%$; $\eta_C=42,6\%$.
		d) $\Delta S_{\mathrm{sis}} = 0$; $\Delta S_{\mathrm{med}}=1,26$\,kJ/K;
		$\Delta S_U=1,26$\,kJ/K $>0 \to$ irreversible. 

	\item {\bf{Entropía del hielo}}

		Calcule el cambio de entropía de un bloque de hielo de $54.0$\,g que se
		encuentra a $250$\,K cuando es convertido de manera reversible y a
		presión constante, en vapor a $390$\,K.
        \\{\bf{R}}: $\Delta S=(10+66+70,45+326,8+5)$\,J/K$=478,25$\,J/K.
	

	\item {\bf{Ciclo de Stirling}}

		Cien moles de un gas ideal biatómico es sometido a un ciclo de Stirling
		internamente reversible. El gas, inicialmente a una presión
		$p_A=150$\,kPa y $T_A=300$\,K, experimenta una compresión isoterma
		hasta alcanzar el estado $B$, seguido de una compresión isócora hasta
		una presión $p_C=3$\,MPa y $T_C=2000$\,K. Luego se realiza una
		expansión isotérmica hasta alcanzar el estado $D$ tal que $V_D=V_A$. El
		ciclo se cierra con una decompresión isócora hasta volver al estado
		$A$.
		\begin{enumerate}
			\item Grafique el ciclo en un diagrama $PV$ y en un diagrama $TS$.
			\item Complete el cuadro de estados y el cuadro de
				transformaciones, calculando además los cambios de entropía del
				gas en cada transformación.
			\item Calcule la variación neta de entropía del Universo
			\item Calcule el rendimiento del ciclo y compárelo con el de un
				ciclo de Carnot equivalente. Analice el resultado en función
				del teorema de Carnot.
		\end{enumerate}
        {\bf{R}}: b) 
		$P_A=150000$\,Pa, $V_A=1,663$\,m$^3$, $n_A=100$\,mol, $T_A=300$\,K; 
		$P_B=450000$\,Pa, $V_B=0,554$\,m$^3$, $n_B=100$\,mol, $T_B=300$\,K; 
		$P_C=3000000$\,Pa, $V_C=0,554$\,m$^3$, $n_C=100$\,mol, $T_C=2000$\,K; 
		$P_C=1000000$\,Pa, $V_C=1,663$\,m$^3$, $n_C=100$\,mol, $T_C=2000$\,K; 
		$Q_1=-274016$\,J, $\Delta U_1=0$, $W_1=-274016$\,J, $\Delta S_1=-913,4$\,J/K.
		$Q_2=3553450$\,J, $\Delta U_2=3533450$\,J, $W_2=0$, $\Delta S_2=3943,2$\,J/K.
		$Q_3=1826773$\,J, $\Delta U_3=0$, $W_3=1826773$\,J, $\Delta S_3=913,4$\,J/K.
		$Q_4=-3553450$\,J, $\Delta U_2=-3533450$\,J, $W_2=0$, $\Delta S_2=-3943,2$\,J/K.
		c) $\Delta S_{\mathrm{sis}} = 0$; $\Delta S_{\mathrm{med}}=10011$\,J/K;
		$\Delta S_U=10011$\,J/K $>0 \to$ irreversible. 
		d) $\eta= 0,29 = 29\%$; $\eta_C=85\%$.

	\item {\bf{Ciclo Otto}}

		Dibuje el ciclo Otto del problema 36 (guía 04) en un diagrama $TS$
		identificando cada transformación y los cambios de energía con el
		medio.

	\item {\bf{Verificando el segundo principio}}

		Compruebe el cumplimiento del segundo principio de la termodinámica en
		para los problemas 26, 27, 28, 29, 30 y 36, verificando que la entropía total
		del universo no decrece en ninguno de esos casos. Identifique los
		ciclos reversibles y los ciclos irreversibles.
        \\{\bf{R}}: trabajo personal individual.
		
	
	\item {\bf{Aumento de entropía, II}}
		
		Un gas ideal biatómico se encuentra en el interior de un cilindro de
		paredes adiabáticas. El estado inicial del gas: $P_A=101325$\,Pa,
		$T_A=293$\,K y $V_A=100$\,cm$^3$. La tapa del cilindro es un pistón
		móvil también adiabático. De manera brusca se coloca una pesa en el
		pistón y la presión en el interior aumenta a $P_B=2 P_A$. Para hacerlo
		debe tener en cuenta que si bien la transformación es adiabática, esta
		no es reservible, y por lo tanto no es posible aplicar
		$pV^{\gamma}=$cte. Entonces:
		\begin{enumerate}
			\item Verifique que la temperatura final está dada por $T_B=\left (
				\frac{r (\gamma-1) + 1}{\gamma}\right ) T_A$, donde $r=p_B/p_A$.
			\item Luego, a partir de esta expresión y utilizando la ecuación de
				estado, muestre que el volúmen final en este caso está dado por
				$V_B=\left ( \frac{r (\gamma-1) + 1}{\gamma r}\right ) V_A$.
			\item Determine el estado final del gas, $B$, 
			\item Calcule el trabajo total irreversible realizado sobre el
				sistema. 
			\item Calcule la variación de entropía del sistema, del medio y del
				Universo, y verifique que el proceso fue irreversible.
			\item Fnalmente, compare los resultados obtenidos con los que se
				obtienen durante una compresión adiabática reversible desde
				$P_A$ a $P_B$.
		\end{enumerate}
		{\bf{R}}: 
		c) $n_B=0,00416$\,mol; $P_B=202650$\,Pa; $V_B=64,3$\,cm$^3$; $T_B=376,7$\,K. 
		d) $W_i = p_B (V_B - V_A) = -7,24$\,J; $\Delta U = 7,24$\,J.
		e) $\Delta S_{\mathrm{sis}}=6,45$\,mJ/K; $\Delta S_{\mathrm{med}}=0$
		(adiabático); $\Delta S_U=6,45$\,mJ/K$>0\to$ irreversible.
		f) En el caso reversible, vale $pV^\gamma=$cte, y por lo tanto el
		estado final del gas sería: $n_B=0,00416$\,mol; $P_B=202650$\,Pa;
		$V_B=60,95$\,cm$^3$; $T_B=357$\,K. El trabajo realizado sobre el
		sistema hubiera sido $W_r=-5,55$\,J y por ende $\Delta U=5,55$. Y dado
		que hubiera sido un proceso adibático y reversible, ergo isentrópico,
		$\Delta S=0$.
\end{enumerate}
\end{document}
%%%%
