\documentclass[a4paper,12pt]{article}
\usepackage[spanish]{babel}
\hyphenation{co-rres-pon-dien-te}
\usepackage[utf8]{inputenc}
\usepackage[T1]{fontenc}
\usepackage{graphicx}
\usepackage[pdftex,colorlinks=true, pdfstartview=FitH, linkcolor=blue,
citecolor=blue, urlcolor=blue, pdfpagemode=UseOutlines, pdfauthor={H. Asorey},
pdftitle={F3B+F4B - Guía 03}]{hyperref}
\usepackage[adobe-utopia]{mathdesign}

\hoffset -1.23cm
\textwidth 16.5cm
\voffset -3.5cm
\textheight 26.5cm

%----------------------------------------------------------------
\begin{document}
\title{
{\normalsize{Universidad Nacional de Río Negro - Profesorado de Física}}\\
Física 3B+4A  2018 \\ Guía 03: Primer Principio \\ {\bf{Aún no está lista}}
}
\author{Asorey}
\maketitle

\begin{enumerate}
	\setcounter{enumi}{21}      %% esta guia va del problema 01 al 06

    \item {\bf{Tres cilindros}}

        Tres pistones cilindros idénticos contienen $1$\,mol de un gas ideal 
        monoatómico, biatómico y triatómico respectivamente. Todos los gases se
        encuentran inicialmente en CNPT. Si al gas contenido en cada pistón se
        le entregan $13.2$\,kJ en forma de calor de manera que la presión se
        mantiene constante, calcule el volumen final de cada recipiente.
        ¿Qué tipo de gas usaría si tuviera que hacer un elevador utilizando 
        estos pistones? Justifique en el marco de la teoría cinética de los
        gases los resultados obtenidos.

	\item {\bf{Diferencias}}
		
		Un pistón cilíndrico contiene un mol de un gas ideal monoatómico en
		CNPT.  Partiendo siempre del mismo estado inicial, primero se somete al
		gas a una expansión isobárica, luego a una expansión isotérmica y
		finalmente en una expansión adiabática, hasta alcanzar en los tres
		casos un volumen final que es el doble del volumen inicial. Calcule: a)
		Los volúmenes iniciales y finales en cada proceso. b) Si el recipiente
		es la cámara de un pistón cilíndrico de radio $r=10$\,cm, calcule el
		trabajo realizado y la altura inicial y final del pistón, en cada
		proceso. c) Cuando corresponda calcule para cada proceso: la cantidad
		de calor suministrada, las temperaturas iniciales y finales, y la
		variación en la energía interna del gas.

\end{enumerate}
\end{document}
%%%%
