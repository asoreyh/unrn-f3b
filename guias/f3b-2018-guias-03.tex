\documentclass[a4paper,12pt]{article}
\usepackage[spanish]{babel}
\hyphenation{co-rres-pon-dien-te}
\usepackage[utf8]{inputenc}
\usepackage[T1]{fontenc}
\usepackage{graphicx}
\usepackage[pdftex,colorlinks=true, pdfstartview=FitH, linkcolor=blue,
citecolor=blue, urlcolor=blue, pdfpagemode=UseOutlines, pdfauthor={H. Asorey},
pdftitle={F3B+F4B - Guía 03}]{hyperref}
\usepackage[adobe-utopia]{mathdesign}

\hoffset -1.23cm
\textwidth 16.5cm
\voffset -3.5cm
\textheight 26.5cm

%----------------------------------------------------------------
\begin{document}
\title{
{\normalsize{Universidad Nacional de Río Negro - Profesorado de Física}}\\
Física 3B+4A  2018 \\ Guía 03: Primer Principio
}
\author{Asorey}
\date{20 de Abril de 2018}
\maketitle

\begin{enumerate}
	\setcounter{enumi}{21}      %% esta guia va del problema 01 al 06

    \item {\bf{Tres cilindros}}

        Tres pistones cilindros idénticos contienen $1$\,mol de un gas ideal 
        monoatómico, biatómico y triatómico respectivamente. Todos los gases se
        encuentran inicialmente en CNPT. Si al gas contenido en cada pistón se
        le entregan $13.2$\,kJ en forma de calor de manera que la presión se
        mantiene constante, calcule el volumen final de cada recipiente.
        ¿Qué tipo de gas usaría si tuviera que hacer un elevador utilizando 
        estos pistones? Justifique en el marco de la teoría cinética de los
        gases los resultados obtenidos.
		\\{\bf{R}}: $V_i=0,0224$\,m$^3$; $V_{f,\mathrm{mono}}=0,0745$\,m$^3$;
		$V_{f,\mathrm{bi}}=0,0596$\,m$^3$; $V_{f,\mathrm{tri}}=0,055$\,m$^3$;

	\item {\bf{Diferencias}}
		
		Tres pistones cilíndricos idénticos de radio $r=0.2$\,m contienen cada
		uno 10\,mol de un gas ideal monoatómico a $T=1000$\,K y $p=20$\,atm.
		El gas del primer pistón es sometido a una expansión isobárica, el del
		segundo a una expansión isotérmica y el del tercer cilindro a una
		expansión adiabática. En todos los casos el volumen final es el doble
		del volumen inicial. Calcule: a) Los volúmenes iniciales y finales en
		cada pistón. b) Calcule el trabajo realizado por el gas y la altura
		inicial y final de cada pistón. c) Cuando corresponda calcule para cada
		proceso: la cantidad de calor suministrada, las temperaturas iniciales
		y finales, y la variación en la energía interna del gas.
		\\{\bf{R}}: a) son iguales para todos, $V_i=0,04103$\,m$^3$ y
		$V_{f}=0,08206$\,m$^3$; b) las alturas son iguales para todos los
		pistones $h_i=0,327$\,m y $h_f=0,653$\,m. Los trabajos serán
		$W_1=83,15$\,kJ; $W_2=57,6$\,kJ; y $W_3=46,1$\,kJ; c) $\Delta
		U_1=124,7$\,kJ y $Q_1=208,85$\,kJ; $\Delta U_2=0$ y $Q_2=57,6$\,kJ;
		$\Delta U_3=-46,1$\,kJ y $Q_3=0$;  
	
	\item {\bf{Transformaciones}}
		
		Un mol de un gas ideal a presión $P_A$ ocupa un volumen $V_A$. Se lo
		calienta en una transformación isócora entregándole una cantidad de
		calor $Q_{1}$ hasta que el sistema alcanza la presión
		$P_B$. Luego se vuelve a calentarlo, entregándole una cantidad de calor
		$Q_2 = Q_1$, pero mediante una
		transformación isobárica hasta alcanzar un volumen $V_C$. 
		\begin{enumerate}
			\item En un diagrama $P-V$, dibuje las transformaciones que el gas
				realiza, identificando las curvas isotermas asociadas a cada
				estado. ¿Es un ciclo? ¿Por qué?
			\item Obtenga una expresión del cociente entre los calores
				específicos $C_P$ y $C_V$ como función de las temperaturas
				de los estados.
			\item A partir del valor del cociente de los calores específicos
				para un gas ideal monoatómico, $C_P / C_V = \gamma = 5/3$, y
				sabiendo que inicialmente el gas se encontraba en CNPT y que la
				presión final es el doble de la presión inicial, calcule:
				\begin{enumerate}
					\item El volumen inicial $V_A$ y final $V_C$
					\item La presión final $P_C$
					\item Las temperaturas $T_B$ y $T_C$. 
					\item La cantidad de calor total suministrada.
				\end{enumerate}
		\end{enumerate}
		{\bf{R}}: b) $\gamma = \frac{C_P}{C_V} = \frac{T_B-T_A}{T_C-T_B}$; c1)
		$V_A=V_B=0,0224$\,m$^3$, $V_C=0,0291$\,m$^3$; c2) $P_C=P_B=202650$\,Pa;
		c3) $T_A=273$\,K, $T_B=546$\,K, $T_C=709,8$\,K; c4)
		$Q_1=Q_2=3404,5$\,J.
		
	\item {\bf{Caldera}}
		
		En el recipiente de presión de una caldera hay una temperatura de
		230\,$^\mathrm{o}$C y una presión constante de 30\,bar. En cada ciclo
		de trabajo el vapor desplaza un pistón con una superficie de
		$0.2$\,m$^2$ una distancia de $0.4$\,m. a) ¿Cuánto vale el trabajo
		entregado en cada ciclo? b) ¿Cuál es la potencia entregada por la
		máquina de vapor cuando se desarrollan 600 ciclos de trabajo por
		minuto?
		\\{\bf{R}}: a) $W=240$\,kJ; b) La potencia erogada es $P=2,4$\,MW.
	
	\item {\bf{Un ciclo para no perder la costumbre}}
		
		Una máquina térmica utiliza como fluido un gas ideal monoatómico, y
		funciona con dos fuentes a temperaturas $T_A = 297$\,K y $T_B =
		990$\,K. El ciclo consiste en un calentamiento isocórico, seguido por
		una expansión isotérmica, para terminar con una compresión isobárica.
		El volumen inicial es $V_A=0.1$\,m$^3$ a una presión de
		$P_A=101325$\,Pa.
		
		\begin{enumerate}
			\item Dibuje el ciclo en un diagrama P-V.
			\item Completar el cuadro de estados 
			\item Completar el cuadro de transformaciones
			\item Hallar el rendimiento $\eta$ del ciclo, y compararlo con el
				rendimiento del ciclo de Carnot funcionando entre esas mismas
				temperaturas.
			\item Si el motor opera a un régimen de $3000$ ciclos por segundo,
				calcule la potencia del motor y la cantidad de calor entregada
				por segundo a la fuente fría.
		\end{enumerate}
		{\bf{R}}: b) $P_A=101325$\,Pa, $V_A=0,1$\,m$^3$, $n_A=4,1035$\,mol,
		$T_A=297$\,K; $P_B=337750$\,Pa, $V_A=0,1$\,m$^3$, $n_A=4,1035$\,mol,
		$T_A=990$\,K; $P_C=101325$\,Pa, $V_A=0,33$\,m$^3$, $n_A=4,1035$\,mol,
		$T_A=990$\,K; c) $Q_1=35463,8$\,J, $\Delta U_1=35463,8$\,J, $W_1=0$;
		$Q_2=40664,2$\,J, $\Delta U_2=0$\,J, $W_2=40664,2$\,J;
		$Q_3=-59106,3$\,J, $\Delta U_3=-35463,8$\,J, $W_3=-23642,5$\,J; d)
		$\eta=17021,7/76127,9 = 0,2236 = 22,36\%$; $\eta_{\mathrm{Carnot}} =
		0,7 = 70\%$; e) Potencia $P=51,06$\,MW,
		$Q_{\mathrm{fria}}=Q_3=-177,32$\,J cada segundo.
	
	\item {\bf{El cuadrado}}
		
		Una máquina térmica está equipada con $n=1000$\,moles de un gas ideal
		di-atómico, inicialmente en CNPT, que opera con el siguiente ciclo: 1)
		calentamiento isocórico hasta quintuplicar la temperatura inicial; 2)
		expansión isobárica hasta quintuplicar el volumen inicial; 3)
		enfriamiento isocórico; 4) compresión isobárica.
		
		\begin{enumerate}
			\item Complete el cuadro de estados, encontrando los valores de
				$P$, $V$, $T$ y $n$ para cada uno de los estados $A$, $B$, $C$
				y $D$. 
			\item En el diagrama $P-V$ ubique los estados y dibuje las
				transformaciones experimentadas por el gas.
			\item Complete el cuadro de transformaciones, encontrado los
				cambios de energía interna, calor y trabajo en cada
				transformación.
			\item Calcule el rendimiento de la máquina y compárelo con el
				rendimiento del ciclo de Carnot equivalente (aquel que funciona
				con las mismas fuentes térmicas).
		\end{enumerate}
		{\bf{R}}: a) $P_A=101325$\,Pa, $V_A=22,4$\,m$^3$, $n_A=1000$\,mol,
		$T_A=273$\,K; $P_B=506634,4$\,Pa, $V_B=22,4$\,m$^3$, $n_B=1000$\,mol,
		$T_B=1365$\,K; $P_C=506634,4$\,Pa, $V_C=112$\,m$^3$, $n_C=1000$\,mol,
		$T_C=6825$\,K; $P_D=101325$\,Pa, $V_D=112$\,m$^3$, $n_D=1000$\,mol,
		$T_D=1365$\,K; c) $Q_1=22,7$\,MJ, $\Delta U_1=22,7$\,MJ, $W_1=0$;
		$Q_2=158,9$\,MJ, $\Delta U_2=113,5$\,MJ, $W_2=45,4$\,MJ;
		$Q_3=-113,5$\,MJ, $\Delta U_3=-113,5$\,MJ, $W_3=0$; $Q_4=-31,8$\,MJ,
		$\Delta U_4=-22,7$\,MJ, $W_4=-9,1$\,MJ; d) $\eta=36,3/181,6 = 0,2 =
		20\%$; $\eta_{\mathrm{Carnot}} = 0,96 = 96\%$;

	\item {\bf{Carnot}}

		Una máquina térmica funciona con $n=0,2$\,mol de un gas ideal biatómico
		siguiendo un ciclo de Carnot entre las temperaturas
		$T_{\mathrm{caliente}}=500$\,K y $T_{\mathrm{fria}}=300$\,K. La presión
		del estado inicial es $P_A=10^6$\,Pa y luego de la primera expansión
		isotérmica el volumen se duplica, es decir, $V_B = 2 V_A$.
		\begin{enumerate}
			\item Complete el cuadro de estados, encontrando los valores de
				$P$, $V$, $T$ y $n$ para cada uno de los estados $A$, $B$, $C$
				y $D$. 
			\item En el diagrama $P-V$ ubique los estados y dibuje, en escala,
				las transformaciones experimentadas por el gas.
			\item Complete el cuadro de transformaciones, encontrado los
				cambios de energía interna, calor y trabajo en cada
				transformación.
			\item Calcule el eficiencia de la máquina a partir de la definición
				$\eta=W_{\mathrm{neto}} / Q_{>0}$ y compárelo con el obtenido
				utilizando la fórmula del rendimiento de la máquina de Carnot. 
		\end{enumerate}
		{\bf{R}}: a) 
		$P_A=10^6$\,Pa, $V_A=8,31\times10^{-4}$\,m$^3$, $n_A=0,2$\,mol,
		$T_A=500$\,K; $P_B=5\times 10^5$\,Pa, $V_B=16,63\times10^{-4}$\,m$^3$,
		$n_B=0,2$\,mol, $T_B=500$\,K; $P_C=83656,4$\,Pa,
		$V_C=59,63\times10^{-4}$\,m$^3$, $n_C=0,2$\,mol, $T_C=300$\,K;
		$P_D=167312,9$\,Pa, $V_D=29,81\times10^{-4}$\,m$^3$, $n_D=0,2$\,mol,
		$T_D=300$\,K; c) $Q_1=576,3$\,J, $\Delta U_1=0$, $W_1=576,3$\,J;
		$Q_2=0$, $\Delta U_2=-831,4$\,J, $W_2=831,4$\,J; $Q_3=-345,8$\,J,
		$\Delta U_3=0$, $W_3=-345,8$\,J; $Q_4=0$, $\Delta U_4=831,4$\,J,
		$W_3=-831,4$\,J; d) $\eta=230,5/576,3 = 0,4 = 40\%$;
		$\eta_{\mathrm{Carnot}} = 0,4 = 40\%$.
\end{enumerate}
\end{document}
%%%%
