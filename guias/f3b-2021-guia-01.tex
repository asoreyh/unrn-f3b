\documentclass[a4paper,12pt]{article}
\usepackage[spanish]{babel}
\hyphenation{co-rres-pon-dien-te}
\usepackage[utf8]{inputenc}
\usepackage[T1]{fontenc}
\usepackage{graphicx}
\usepackage[pdftex,colorlinks=true, pdfstartview=FitH, linkcolor=blue,
citecolor=blue, urlcolor=blue, pdfpagemode=UseOutlines, pdfauthor={H. Asorey},
pdftitle={F3B - Guía 01}]{hyperref}
\usepackage[adobe-utopia]{mathdesign}

\hoffset -1.23cm
\textwidth 16.5cm
\voffset -2.5cm
\textheight 26.0cm

%----------------------------------------------------------------
\begin{document}
\title{
{\normalsize{Universidad Nacional de Río Negro - Profesorado de Física}}\\
Física 3B  2021 \\ Guía 01: El Calor}
\author{Asorey - Calderón}
\date{09 de Marzo de 2021}
\maketitle

\begin{enumerate}
	\setcounter{enumi}{0}

	\item {\bf{Goma de auto}}
		
		Un neumático de auto tiene una sobrepresión, respecto de la presión
		atmosférica ($p=1$\,atm), de $2000$\,hPa.
		\begin{enumerate}
			\item ¿Cuánto vale esa sobrepresión en las unidades técnicas usuales en
				Argentina? ¿y en bares? ¿y en p.s.i.\footnote{p.s.i. es una
				unidad imperial de presión y es la abreviatura de libras por
				pulgada cuadrada, por sus siglas en inglés para {\textit{pounds
				per square inch}}}? 
				
				{\bf R}: $p=2000$\,hPa $= 2$\,bar $= 29,01$\,psi $= 1,974$\,atm.
			\item La superficie total de contacto entre los cuatro neumáticos y
				el pavimento es de $0,05$\,m$^2$. ¿Cual es la masa del
				automóvil?
				
				{\bf R} $P = 10$\,kN, $m=1020,4$\,kg
		\end{enumerate}
	
	\item {\bf{No desinfle lo inflado}}
		
		En una fría mañana invernal (temperatura = $0^\mathrm{o}$\,C) la
		presión en un neumático es de $2000$\,hPa. Luego de un viaje a alta
		velocidad la temperatura del aire en el interior del neumático es de
		$323$\,K.
		\begin{enumerate}
			\item ¿Qué tipo de transformación experimentó el gas?
			\item ¿Cuál es la sobrepresión a esa temperatura?

				{\bf R}: $p=2366,3$\,hPa.
			\item Si en ese instante, el propietario del auto desinfla
				rápidamente los neumáticos para volver a la presión recomendada
				sin esperar a que éstas se enfríen, ¿cuál será la presión final
				cuando las cubiertas vuelvan a la temperatura del ambiente en
				ese momento?
				{\bf R}: $p=1690,4$\,hPa.
		\end{enumerate}
	
	\item {\bf{El tubo}}
		
		Un tubo cilíndrico de acero con un diámetro interior de $0.2$\,m y un
		altura de $1$\,m contiene nitrógeno a una presión de $10^5$\,hPa y
		$T=273$\,K.

		\begin{enumerate}
			\item ¿Cuál es la masa total del gas? La masa molar del nitrógeno
				es $0.028$\,kg\,mol$^{-1}$
				
				{\bf R}: $n=138,3$\,mol, entonces $m=3,87$\,kg.
			\item ¿Qué fuerza ejerce el gas sobre la superficie interior del tubo?
				
				{\bf R}: $S=0.691$\,m$^2$, entonces $F=6,911 \times 10^6$\,N.
			\item ¿Qué presión habrá en el interior del tubo si, a temperatura
				constante, se dejara escapar la mitad del gas por un válvula?
				
				{\bf R}: $p=5 \times 10^4$\,hPa.
		\end{enumerate}
 
	\item {\bf{Conectando recipientes}}

		Un recipiente de $1$\,L ($0.001$\,m$^3$) lleno de un gas ideal a una
		presión de $100$\,kPa se conecta con otro recipiente de $0.003$\,m$^3$
		conteniendo un gas ideal a una presión de $50$\,kPa. Suponiendo que
		ambos recipientes están en contacto con un baño térmico, calcule la
		presión final del sistema una vez que los recipientes se conectan.

		{\bf R}: $p_f = 62,5$\,kPa.
	\item {\bf{Conectando recipientes distintos}}
	
		Dos recipientes están unidos por un tubo de volumen despreciable con
		una válvula en el tubo y que inicialmente se encuentra abierta. Uno de
		ellos tiene un volumen cinco veces mayor que el otro. Todo el sistema
		está lleno de aire (masa molar $M=29$\,g\,mol$^{-1}$) a una presión de
		$1866.5$\,hPa y una temperatura de $293$\,K. Luego se cierra la válvula
		y se procede a calentar el recipiente grande hasta una temperatura de
		$373$\,K, manteniendo el recipiente pequeño a la temperatura inicial.
		¿Cuál es la presión final del sistema luego de abrirse la válvula y
		conectar ambos recipientes? 
	
		{\bf R}: $T_f = 359,67$\,K, entonces $p_f = 2291,2$\,hPa.
	\item {\bf{Globo meteorológico}}
		
		Un balón meteorológico esférico es rellenado con Helio al nivel del mar
		(CNPT)\footnote{CNPT es la abreviatura para {\textit{Condiciones
		Normales de Presión y Temperatura}} y corresponde a la presión
		atmosférica de referencia, $P=101325$\,Pa, a una temperatura de
		$T=273.15$\,K. Verifique que en estas condiciones, el volumen ocupado
		por 1 mol de un gas ideal es $0.0224$\,m$^3$ ($22.4$\,L).}.  Cuando
		está listo para iniciar su ascenso, tiene un radio de $2$\,m.  Sabiendo
		que la presión atmosférica $p$ (medida en hPa) como función de la
		altura $h$ obedece la siguiente ley, \[ p = 1013.2 \exp \left (
		-\frac{mgh}{RT} \right ),\] donde $M$ es la masa molar media del aire
		($m=29$\,g\,mol$^{-1}$). A medida que asciende, el globo aumenta su
		volumen hasta que alcanza un tamaño máximo y luego estalla. 
		\begin{enumerate}
			\item Calcule la cantidad de Helio en moles que se usó para llenar
				el globo.
				\\{\bf R}: $V_i=33,51$\,m$^3$, entonces $n=1496$\,mol.
			\item Calcule el empuje (en newtons) que tendrá el globo al iniciar
				su ascenso. 
				\\{\bf R}: Igual al peso del volúmen de aire desalojado, en
				CNPT $E=425.2$\,N.
			\item Calcule la altura a la que se encuentra el globo cuando su
				radio es de $r=3$\,m.
				\\{\bf R}: $h=9715$\,m.
			\item Calcule el radio del globo justo antes de estallar a
				$h=27$\,km de altura.
				\\{\bf R}: $h=27000$\,m, $V_f=1062$\,m$^3$, $r_f=6,33$\,m.
		\end{enumerate}

	\item {\bf{Teoría cinética}}

		Consideremos una determinada cantidad de Helio contenido en un
		recipiente esférico y rígido en CNPT.
		\begin{enumerate}
			\item ¿Qué cantidad de moles y de átomos hay en $1$\,m$^3$ de
				helio en estas condiciones?
				\\{\bf R}: $n=44,64$\,mol, $N=2.6883\times 10^{25}$\,átomos.
			\item Suponiendo que el radio de un átomo de Helio puede
				aproximarse por una esfera de $r=2 a_0$, donde $a_0$ es el
				radio de Bohr, calcule el volumen total ocupado por los átomos
				en el gas y la fracción de volumen que estos ocupan del volumen
				total. 
				\\{\bf R}: $V_{\mathrm{at}}=1,335\times 10^{-4}$\,m$^3$.
				entonces $f_{\mathrm{at}}=1,335\times 10^{-4}$.
			\item Calcule la energía cinética media y la velocidad media de un
				átomo de Helio en esas condiciones. 
				\\{\bf R}: $v_{\mathrm{RMS}}=1304,4$\,m/s.
			\item Estime la cantidad media de colisiones por segundo que se
				producen en las paredes del recipiente. 
				\\{\bf R}: Esfera de $1$\,m$^3$, $r=0,62$\,m, $S=4,84$\,m$^2$,
				entonces $N_C=1,7 \times 10^{29}$\,colisiones/s sobre toda la
				superficie de la esfera.
		\end{enumerate}
	
	\item {\bf{Alta presión}}
	
		Un recipiente contiene $201,8$\,kg de Neón a una presión de $500$\,bares. 
		\begin{enumerate}
			\item ¿Cuál debería ser el volumen del recipiente para que la
				velocidad media de las moléculas sea igual a la velocidad de
				escape terrestre? 
				\\{\bf R}: $v_e=11180$\,m/s, entonces $T=101100$\,K, y como
				$n=10000$\,mol y $p=500$\,bar, entonces $V=168,12$\,m$^3$. 
			\item En esas condiciones, calcule la energía cinética media y el
				número de colisiones por segundo que se produce con las paredes
				del recipiente suponiendo que el mismo es esférico. 
				\\{\bf R}: $\langle E_K \rangle = 2,094 \times 10^{-18}$\,J
				($v_{\mathrm{RMS}}=11,18$\,km/s), $N_C=5,9 \times
				10^{31}$\,colisiones/s.
			\item ¿Dependerá el resultado anterior de la forma del recipiente?
				Justifique
		\end{enumerate}

	\item {\bf{Júpiter y Marte}}

		La velocidad de escape de Júpiter es de alrededor de $v_e=60$\,km/s y
		su temperatura superficial es $T=-150\,^\circ$C. Calcule la velocidad
		RMS para a) H$_2$; b) O$_2$; y c) CO$_2$ a esa temperatura. Saque sus
		conclusiones y diga si es probable encontrar esos gases en la atmósfera
		de Júpiter. Luego, repita sus cálculos para Marte, con $v_e=5$\,km/s y
		$T=0\,^\circ$C.
		\\{\bf R}: Júpiter: a) $v_{\mathrm{RMS}}=1235$\,m/s; b)
		$v_{\mathrm{RMS}}=309.9$\,m/s; c) $v_{\mathrm{RMS}}=264.3$\,m/s;
		\\{\bf R}: Marte: a) $v_{\mathrm{RMS}}=1839$\,m/s; b)
		$v_{\mathrm{RMS}}=461.5$\,m/s; c) $v_{\mathrm{RMS}}=393.5$\,m/s;
	
	\item {\bf{Gas monoatómico}}
	
		Se dispone de una determinada cantidad de gas ideal monoatómico
		almacenado en un recipiente rígido de $0.04$\,m$^3$ a temperatura
		ambiente ($293$\,K) y con una presión de $20265$\,kPa. a) Calcule la
		cantidad de gas contenida, medida en moles, y determine el número de
		moléculas y de átomos contenidos en el interior del recipiente. b) Uno
		de los operarios de la planta enciende involuntariamente un fuego cerca
		del recipiente. La temperatura del mismo aumenta hasta alcanzar los
		$423$\,K, momento en el cual la válvula de seguridad se activa y deja
		escapar parte del gas almacenado, hasta que la presión vuelve a ser la
		presión de trabajo ($20265$\,kPa) a esa temperatura. Calcule b1) la
		presión a la cual se activó la válvula; b2) la cantidad de gas
		remanente luego del escape; b3) la energía interna total del gas en el
		recipiente en cada uno de los siguientes momentos: estado inicial;
		inmediatamente antes que se active la válvula de seguridad; cuando se
		recupera la presión de trabajo.
		\\{\bf R}: a) $n=333$\,mol; $N=2\times10^{26}$\,átomos; b1)
		$p=29300$\,kPa; b2) $n_i=333$\,mol; $n_f=230$\,mol; b3)
		$U_1=1.217$\,MJ, $U_2=1.757$\,MJ, $U_3=1.214$\,MJ.

	\item {\bf{Trabajo sobre el gas}}

		Realizando un trabajo de $100$\,J sobre un gas ideal es posible
		comprimirlo en forma isotérmica hasta que alcanza un volumen igual a la
		quinta parte del volumen inicial, $V_f = 1/5 V_i$. Calcule: a) El
		cambio de energía interna del gas; y b) La cantidad de calor
		intercambiada con el medio durante el proceso. 
		\\{\bf R}: a) $\Delta U = 0$; b) $Q=-100$\,J.
	
	\item {\bf{Expansión isobárica}}
		
		a) Calcule la variación de temperatura que experimentan 2 moles de un
		gas monoatómico ideal si se le permiten expandirse en forma isobárica
		al suministrarle $3000$\,J en forma de calor. b) ¿Que hubiera pasado si
		el gas hubiera sido biatómico? c) ¿y triatómico?
		\\{\bf R}: a) $\Delta T = +72,16$\,K; b) $\Delta T = +51,54$\,K; c)
		$\Delta T = +45,10$\,K;

	\item {\bf{Tres gases}}

		Tres recipientes rígidos e idénticos contienen en CNPT $2$\,mol de un gas ideal
		monoatómico, biatómico y triatómico respectivamente. Si al gas de cada
		recipiente se le entregan $30$\,kJ de energía en forma de calor: a)
		calcule la temperatura final y el cambio de energía interna del gas en
		cada recipiente. b) Compare los resultados y justifique utilizando la
		teoría cinética de los gases.  
		\\{\bf R}: a) $\Delta T_1 = +1202,7$\,K; $\Delta T_2 = +721,6$\,K;
		$\Delta T_3 = +601,4$\,K; $\Delta U_1 = +30$\,kJ; $\Delta U_2 =
		+30$\,kJ; $\Delta U_3 = +30$\,kJ.

	\item {\bf{Expansión isoterma}}

		Tres pistones idénticos contienen cada uno $2$\,mol de un gas ideal
		monoatómico, biatómico y triatómico respectivamente. Los gases se
		encuentran inicialmente en CNPT. Mediante la entrega de energía en
		forma de calor se logra que cada uno de los pistones duplique su
		volumen de manera que la temperatura permaneció constante. Calcule para
		cada pistón: a) el estado final de cada gas ($P$, $T$, $n$ y $V$); b)
		la cantidad de calor entregada en cada caso; c) el cambio de energía
		interna del gas; d) el trabajo realizado (ayuda: use la conservación de
		la energía para este caso en su forma $Q=\Delta U + W$) 
		\\{\bf R}: a) mono: $n=2$\,mol; $T=273$\,K; $V=0,0896$\,m$^3$;
		$P=50662,5$\,Pa; bi: $n=2$\,mol; $T=273$\,K; $V=0,0896$\,m$^3$;
		$P=50662,5$\,Pa; tri: $n=2$\,mol; $T=273$\,K; $V=0,0896$\,m$^3$;
		$P=50662,5$\,Pa; b) $Q=W$ entonces, $Q_1 = Q_2 = Q_3 = 3146,7$\,J; c)
		$\Delta U_1 = \Delta U_2 = \Delta U_3 = 0$; d) $Q=W$ entonces, $W_1 =
		W_2 = W_3 = 3146,7$\,J.

	\item {\bf{Pistón oscilante}}

		Un cilindro contiene $n=0.1$\,mol de un gas ideal monoatómico en CNPT.
		El cilindro está sellado en su parte superior por un pistón de
		$m=1.4$\,kg equipado con un sello sin fricción y está en trabado a una
		altura de $h=2.4$\,m. El cilindro está rodeado por aire en CNPT. Una
		vez liberado, el cilindro comienza a caer y una vez que se detiene el
		movimiento, se encuentra en equilibrio con el aire térmico circundante.
		a) Encuentre la nueva altura del pistón. b) Imagine ahora que el pistón
		es ligeramente empujado hacia abajo y luego es soltado. Suponiendo que
		la temperatura del gas no cambia, calcule la frecuencia de oscilación
		del pistón.
		\\{\bf R}: a) $h_f=2.1$\,m. Algunos resultados intermedios: superficie
		base: $S=9,334 \times 10^{-4}$\,m$^2$; radio del cilindro
		$r=17,24$\,mm; presión final $p_f=116034$\,Pa; volumen final
		$V_f=1,96$\,L; b) $\omega=\sqrt{\frac{p_f S}{m h_f}}$, entonces
		frecuencia angular $\omega=6,07$\,rad/s; frecuencia de oscilación
		$f=0,966$\,Hz; periodo de oscilación $\tau = 1,04$\,s. 
		
	\item {\bf{Calores específicos}}
 
		Sean dos objetos, A y B, con una relación de masas $m_A = 2 m_B$ y
		calores específicos $C_A = 2 C_B$. Si a ambos objetos se les entrega la
		misma cantidad de calor $Q$, ¿cómo serán los correspondientes cambios
		en temperatura? a) $\Delta T_A = 4 \Delta T_B$? b) $\Delta T_A = 2
		\Delta T_B$? c) $\Delta T_A = \Delta T_B$? d) $\Delta T_A = 1/2 \Delta
		T_B$? ó e) $\Delta T_A = 1/4 \Delta T_B$.
		\\{\bf R}: e) $\Delta T_A = 1/4 \Delta T_B$.

	\item {\bf{Capacidad calorífica}}

		El metal A es más denso que el metal B. ¿En cuál de ellos espera que la
		capacidad calorífica por unidad de masa sea mayor? Justifique.

	\item {\bf{Pérdidas de calor}} 

		Una casa típica contiene aproximadamente $10^5$\,kg de concreto, con un
		capacidad calorífica específico de $1$\,kJ/(kg K). ¿Cuánto se libera a
		la atmosfera durante la noche cuando su temperatura baja de
		$25\,^\circ$C a $20\,^\circ$C.
		\\{\bf R}: $Q=-5 \times 10^5$\,kJ. 

	\item {\bf{Calentando hielo}}

		¿Qué cantidad de calor se necesita para calentar $60$\,g de hielo
		originalmente a $-10\,^\circ$C para transformarlo en $60$\,g de vapor
		de agua a $140\,^\circ$C. Discrimine el total de calor en sensible y
		latente.
		\\{\bf R}: $Q=187255$\,J; $Q_S=31333$\,J; $Q_L=155922$\,J.

	\item {\bf{Enfriando la bebida}}

		Un vaso de vidrio de $25$\,g de masa contiene 200\,mL de agua a
		$24\,^\circ$C. Si dos cubos de hielo con una masa de $15$\,g cada uno
		originalmente a $-3\,^\circ$C se agregan al agua. ¿Cuál es la
		temperatura final del conjunto?
		\\{\bf R}: Despreciando al vaso, $T_f=283$\,K.

	\item {\bf{Trabajo y calor}}

		Sea $1$\,mol de un gas ideal monoatómico a $p=1$\,atm ocupando un volumen de
		$25$\,L. La energía interna total del gas es $U_A=3800$\,J. El gas es
		calentado mediante una transformación que lo deja en su
		estado final $B$ a una presión de $1$\,atm, un volumen de $75$\,L y la
		energía interna total es de $U_B=11400$\,J. Calcule el trabajo realizado y el
		calor total absorbido por el gas.
		\\{\bf R}: $W=+5066$\,J; $Q=12666$\,J.
\end{enumerate}
\end{document}
%%%%
