\documentclass[a4paper,12pt]{article}
\usepackage[spanish]{babel}
\hyphenation{co-rres-pon-dien-te}
\usepackage[utf8]{inputenc}
\usepackage[T1]{fontenc}
\usepackage{graphicx}
\usepackage[pdftex,colorlinks=true, pdfstartview=FitH, linkcolor=blue,
citecolor=blue, urlcolor=blue, pdfpagemode=UseOutlines, pdfauthor={H. Asorey},
pdftitle={Física 3B}]{hyperref}
\usepackage[adobe-utopia]{mathdesign}

\hoffset -1.23cm
\textwidth 16.5cm
\voffset -2.5cm
\textheight 26.0cm

%----------------------------------------------------------------
\begin{document}
\title{
{\normalsize{Universidad Nacional de Río Negro - Profesorado de Física}}\\
Física 3B+4A  2018 \\ Guía 01: Gases}
\author{Asorey}
\maketitle

\begin{enumerate}
	\setcounter{enumi}{0}      %% esta guia va del problema 01 al 06

	\item {\bf{Goma de auto}}
		
		Un neumático de auto tiene una sobrepresión, respecto de la presión
		atmosférica ($p=1$\,atm), de $2000$\,hPa.
		\begin{enumerate}
			\item ¿Cuánto vale esa sobrepresión en las unidades técnicas usuales en
				Argentina? ¿y en bares? ¿y en p.s.i.\footnote{p.s.i. es una
				unidad imperial de presión y es la abreviatura de libras por
				pulgada cuadrada, por sus siglas en inglés para {\textit{pounds
				per square inch}}}? 
			\item La superficie total de contacto entre los cuatro neumáticos y
				el pavimento es de $0,05$\,m$^2$. ¿Cual es la masa del
				automóvil?
		\end{enumerate}
	
	\item {\bf{No desinfle lo inflado}}
		
		En una fría mañana invernal (temperatura = $0^\mathrm{o}$\,C) la
		presión en un neumático es de $2000$\,hPa. Luego de un viaje a alta
		velocidad la temperatura del aire en el interior del neumático es de
		$323$\,K.
		\begin{enumerate}
			\item ¿Qué tipo de transformación experimentó el gas?
			\item ¿Cuál es la sobrepresión a esa temperatura?
			\item Si en ese momento, el propietario del auto desinfla las gomas
				sin esperar a que éstas se enfríen, ¿cuál será la presión final
				cuando las cubiertas vuelvan a la temperatura del ambiente en
				ese momento?
		\end{enumerate}
	
	\item {\bf{El tubo}}
		
		Un tubo cilíndrico de acero con un diámetro interior de $0.2$\,m y un
		altura de $1$\,m contiene nitrógeno a una presión de $10^5$\,hPa y
		$T=273$\,K.

		\begin{enumerate}
			\item ¿Cuál es la masa total del gas? La masa molar del nitrógeno
				es $0.028$\,kg\,mol$^{-1}$
			\item ¿Qué fuerza ejerce el gas sobre la superficie interior del tubo?
			\item ¿Qué presión habrá en el interior del tubo si, a temperatura
				constante, se dejara escapar la mitad del gas por un válvula?
		\end{enumerate}
 
	\item {\bf{Conectando recipientes}}

		Un recipiente de $1$\,L ($0.001$\,m$^3$) lleno de un gas ideal a una
		presión de $100$\,kPa se conecta con otro recipiente de $0.003$\,m$^3$
		conteniendo un gas ideal a una presión de $50$\,kPa. Suponiendo que
		ambos recipientes están en contacto con un baño térmico, calcule la
		presión final del sistema una vez que los recipientes se conectan.
	
	\item {\bf{Conectando recipientes distintos}}
	
		Dos recipientes están unidos por un tubo de volumen despreciable con
		una válvula en el tubo y que inicialmente se encuentra abierta. Uno de
		ellos tiene un volumen cinco veces mayor que el otro. Todo el sistema
		está lleno de aire (masa molar $M=29$\,g\,mol$^{-1}$) a una presión de
		$1866.5$\,hPa y una temperatura de $293$\,K. Luego se cierra la válvula
		y se procede a calentar el recipiente grande hasta una temperatura de
		$373$\,K, manteniendo el recipiente pequeño a la temperatura inicial.
		¿Cuál es la presión final del sistema luego de abrirse la válvula y
		conectar ambos recipientes? 
	
	\item {\bf{Globo meteorológico}}
		
		Un balón meteorológico esférico es rellenado con Helio al nivel del mar
		(CNPT)\footnote{CNPT es la abreviatura para {\textit{Condiciones
		Normales de Presión y Temperatura}} y corresponde a la presión
		atmosférica de referencia, $P=101325$\,Pa, a una temperatura de
		$T=273.15$\,K. Verifique que en estas condiciones, el volumen ocupado
		por 1 mol de un gas ideal es $0.0224$\,m$^3$ ($22.4$\,L).}.  Cuando
		está listo para iniciar su ascenso, tiene un radio de $2$\,m.  Sabiendo
		que la presión atmosférica $p$ (medida en hPa) como función de la
		altura $h$ obedece la siguiente ley, \[ p = 1013.2 \exp \left (
		-\frac{mgh}{RT} \right ),\] donde $M$ es la masa molar media del aire
		($m=29$\,g\,mol$^{-1}$). A medida que asciende, el globo aumenta su
		volumen hasta que alcanza un tamaño máximo y luego estalla. 
		\begin{enumerate}
			\item Calcule la cantidad de Helio en moles que se usó para llenar el globo.
			\item Calcule el empuje (en newtons) que tendrá el globo al iniciar su ascenso. 
			\item Calcule la altura a la que se encuentra el globo cuando su radio es de $r=3$\,m.
			\item Calcule el radio del globo justo antes de estallar a $h=27$\,km de altura.
		\end{enumerate}

	\item {\bf{Teoría cinética}}

		Consideremos una determinada cantidad de Helio contenido en un recipiente esférico y rígido en CNPT.
		\begin{enumerate}
			\item ¿Qué cantidad de moles y de átomos hay en $1$\,m$^3$ de
				helio en estas condiciones?
			\item Suponiendo que el radio de un átomo de Helio puede
				aproximarse por una esfera de $r=2 a_0$, donde $a_0$ es el
				radio de Bohr, calcule el volumen total ocupado por los átomos
				en el gas y la fracción de volumen que estos ocupan del volumen
				total. 
			\item Calcule la energía cinética media y la velocidad media de un átomo de Helio en esas condiciones. 
			\item Estime la cantidad media de colisiones por segundo que se producen en las paredes del recipiente. 
		\end{enumerate}
	
	\item {\bf{Alta presión}}
	
		Un recipiente contiene $201,8$\,kg de Neón a una presión de $500$\,bares. 
		\begin{enumerate}
			\item ¿Cuál debería ser el volumen del recipiente para que la velocidad media de las moléculas sea igual a la velocidad de escape terrestre? 
			\item En esas condiciones, calcule la energía cinética media y el número de colisiones por segundo que se produce con las paredes del recipiente suponiendo que el mismo es esférico. 
			\item ¿Dependerá el resultado anterior de la forma del recipiente? Justifique
		\end{enumerate}
\end{enumerate}
\end{document}

%%%%
