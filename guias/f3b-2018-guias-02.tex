\documentclass[a4paper,12pt]{article}
\usepackage[spanish]{babel}
\hyphenation{co-rres-pon-dien-te}
\usepackage[utf8]{inputenc}
\usepackage[T1]{fontenc}
\usepackage{graphicx}
\usepackage[pdftex,colorlinks=true, pdfstartview=FitH, linkcolor=blue,
citecolor=blue, urlcolor=blue, pdfpagemode=UseOutlines, pdfauthor={H. Asorey},
pdftitle={F3B+F4B - Guía 02}]{hyperref}
\usepackage[adobe-utopia]{mathdesign}

\hoffset -1.23cm
\textwidth 16.5cm
\voffset -3.5cm
\textheight 26.5cm

%----------------------------------------------------------------
\begin{document}
\title{
{\normalsize{Universidad Nacional de Río Negro - Profesorado de Física}}\\
Física 3B+4A  2018 \\ Guía 02: Calor}
\author{Asorey}
\date{29 de Marzo de 2018}
\maketitle

\begin{enumerate}
	\setcounter{enumi}{8}      %% esta guia va del problema 01 al 06

	\item {\bf{Júpiter y Marte}}

		La velocidad de escape de Júpiter es de alrededor de $v_e=60$\,km/s y
		su temperatura superficial es $T=-150\,^\circ$C. Calcule la velocidad
		RMS para a) H$_2$; b) O$_2$; y c) CO$_2$ a esa temperatura. Saque sus
		conclusiones y diga si es probable encontrar esos gases en la atmósfera
		de Júpiter. Luego, repita sus cálculos para Marte, con $v_e=5$\,km/s y
		$T=0\,^\circ$C.
		\\{\bf R}: Júpiter: a) $v_{\mathrm{RMS}}=1235$\,m/s; b)
		$v_{\mathrm{RMS}}=309.9$\,m/s; c) $v_{\mathrm{RMS}}=264.3$\,m/s;
		\\{\bf R}: Marte: a) $v_{\mathrm{RMS}}=1839$\,m/s; b)
		$v_{\mathrm{RMS}}=461.5$\,m/s; c) $v_{\mathrm{RMS}}=393.5$\,m/s;
	
	\item {\bf{Gas monoatómico}}
	
		Se dispone de una determinada cantidad de gas ideal monoatómico
		almacenado en un recipiente rígido de $0.04$\,m$^3$ a temperatura
		ambiente ($293$\,K) y con una presión de $20265$\,kPa. a) Calcule la
		cantidad de gas contenida, medida en moles, y determine el número de
		moléculas y de átomos contenidos en el interior del recipiente. b) Uno
		de los operarios de la planta enciende involuntariamente un fuego cerca
		del recipiente. La temperatura del mismo aumenta hasta alcanzar los
		$423$\,K, momento en el cual la válvula de seguridad se activa y deja
		escapar parte del gas almacenado, hasta que la presión vuelve a ser la
		presión de trabajo ($20265$\,kPa) a esa temperatura. Calcule b1) la
		presión a la cual se activó la válvula; b2) la cantidad de gas
		remanente luego del escape; b3) la energía interna total del gas en el
		recipiente en cada uno de los siguientes momentos: estado inicial;
		inmediatamente antes que se active la válvula de seguridad; cuando se
		recupera la presión de trabajo.
		\\{\bf R}: a) $n=333$\,mol; $N=2\times10^{26}$\,átomos; b1)
		$p=29300$\,kPa; b2) $n_i=333$\,mol; $n_f=230$\,mol; b3)
		$U_1=1.217$\,MJ, $U_2=1.757$\,MJ, $U_3=1.214$\,MJ.

	\item {\bf{Trabajo sobre el gas}}

		Realizando un trabajo de $100$\,J sobre un gas ideal es posible
		comprimirlo en forma isotérmica hasta que alcanza un volumen igual a la
		quinta parte del volumen inicial, $V_f = 1/5 V_i$. Calcule: a) El
		cambio de energía interna del gas; y b) La cantidad de calor
		intercambiada con el medio durante el proceso. 
		\\{\bf R}: a) $\Delta U = 0$; b) $Q=-100$\,J.
	
	\item {\bf{Expansión isobárica}}
		
		a) Calcule la variación de temperatura que experimentan 2 moles de un
		gas monoatómico ideal si se le permiten expandirse en forma isobárica
		al suministrarle $3000$\,J en forma de calor. b) ¿Que hubiera pasado si
		el gas hubiera sido biatómico? c) ¿y triatómico?
		\\{\bf R}: a) $\Delta T = +72,16$\,K; b) $\Delta T = +51,54$\,K; c)
		$\Delta T = +45,10$\,K;

	\item {\bf{Tres gases}}

		Tres recipientes rígidos e idénticos contienen en CNPT $2$\,mol de un gas ideal
		monoatómico, biatómico y triatómico respectivamente. Si al gas de cada
		recipiente se le entregan $30$\,kJ de energía en forma de calor: a)
		calcule la temperatura final y el cambio de energía interna del gas en
		cada recipiente. b) Compare los resultados y justifique utilizando la
		teoría cinética de los gases.  
		\\{\bf R}: a) $\Delta T_1 = +1202,7$\,K; $\Delta T_2 = +721,6$\,K;
		$\Delta T_3 = +601,4$\,K; $\Delta U_1 = +30$\,kJ; $\Delta U_2 =
		+30$\,kJ; $\Delta U_3 = +30$\,kJ.

	\item {\bf{Expansión isoterma}}

		Tres pistones idénticos contienen cada uno $2$\,mol de un gas ideal
		monoatómico, biatómico y triatómico respectivamente. Los gases se
		encuentran inicialmente en CNPT. Mediante la entrega de energía en
		forma de calor se logra que cada uno de los pistones duplique su
		volumen de manera que la temperatura permaneció constante. Calcule para
		cada pistón: a) el estado final de cada gas ($P$, $T$, $n$ y $V$); b)
		la cantidad de calor entregada en cada caso; c) el cambio de energía
		interna del gas; d) el trabajo realizado (ayuda: use la conservación de
		la energía para este caso en su forma $Q=\Delta U + W$) 
		\\{\bf R}: a) mono: $n=2$\,mol; $T=273$\,K; $V=0,0448$\,m$^3$;
		$P=50662,5$\,hPa; bi: $n=2$\,mol; $T=273$\,K; $V=0,0448$\,m$^3$;
		$P=50662,5$\,hPa; tri: $n=2$\,mol; $T=273$\,K; $V=0,0448$\,m$^3$;
		$P=50662,5$\,hPa; b) $Q=W$ entonces, $Q_1 = Q_2 = Q_3 = 3146,7$\,J; c)
		$\Delta U_1 = \Delta U_2 = \Delta U_3 = 0$; d) $Q=W$ entonces, $W_1 =
		W_2 = W_3 = 3146,7$\,J.

	\item {\bf{Pistón oscilante}}

		Un cilindro contiene $n=0.1$\,mol de un gas ideal monoatómico en CNPT.
		El cilindro está sellado en su parte superior por un pistón de
		$m=1.4$\,kg equipado con un sello sin fricción y está en trabado a una
		altura de $h=2.4$\,m. El cilindro está rodeado por aire en CNPT. Una
		vez liberado, el cilindro comienza a caer y una vez que se detiene el
		movimiento, se encuentra en equilibrio con el aire térmico circundante.
		a) Encuentre la nueva altura del pistón. b) Imagine ahora que el pistón
		es ligeramente empujado hacia abajo y luego es soltado. Suponiendo que
		la temperatura del gas no cambia, calcule la frecuencia de oscilación
		del pistón.
		\\{\bf R}: a) $h_f=2.1$\,m. Algunos resultados intermedios: superficie
		base: $S=9,334 \times 10^{-4}$\,m$^2$; radio del cilindro
		$r=17,24$\,mm; presión final $p_f=116034$\,Pa; volumen final
		$V_f=1,96$\,L; b) $\omega=\sqrt{\frac{p_f S}{m h_f}}$, entonces
		frecuencia angular $\omega=6,07$\,rad/s; frecuencia de oscilación
		$f=0,966$\,Hz; periodo de oscilación $\tau = 1,04$\,s. 
		
	\item {\bf{Calores específicos}}
 
		Sean dos objetos, A y B, con una relación de masas $m_A = 2 m_B$ y
		calores específicos $C_A = 2 C_B$. Si a ambos objetos se les entrega la
		misma cantidad de calor $Q$, ¿cómo serán los correspondientes cambios
		en temperatura? a) $\Delta T_A = 4 \Delta T_B$? b) $\Delta T_A = 2
		\Delta T_B$? c) $\Delta T_A = \Delta T_B$? d) $\Delta T_A = 1/2 \Delta
		T_B$? ó e) $\Delta T_A = 1/4 \Delta T_B$.
		\\{\bf R}: e) $\Delta T_A = 1/4 \Delta T_B$.

	\item {\bf{Capacidad calorífica}}

		El metal A es más denso que el metal B. ¿En cuál de ellos espera que la
		capacidad calorífica por unidad de masa sea mayor? Justifique.

	\item {\bf{Pérdidas de calor}} 

		Una casa típica contiene aproximadamente $10^5$\,kg de concreto, con un
		capacidad calorífica específico de $1$\,kJ/(kg K). ¿Cuánto se libera a
		la atmosfera durante la noche cuando su temperatura baja de
		$25\,^\circ$C a $20\,^\circ$C.
		\\{\bf R}: $Q=-5 \times 10^5$\,kJ. 

	\item {\bf{Calentando hielo}}

		¿Qué cantidad de calor se necesita para calentar $60$\,g de hielo
		originalmente a $-10\,^\circ$C para transformarlo en $60$\,g de vapor
		de agua a $140\,^\circ$C. Discrimine el total de calor en sensible y
		latente.
		\\{\bf R}: $Q=187255$\,J; $Q_S=31333$\,J; $Q_L=155922$\,J.

	\item {\bf{Enfriando la bebida}}

		Un vaso de vidrio de $25$\,g de masa contiene 200\,mL de agua a
		$24\,^\circ$C. Si dos cubos de hielo con una masa de $15$\,g cada uno
		originalmente a $-3\,^\circ$C se agregan al agua. ¿Cuál es la
		temperatura final del conjunto?
		\\{\bf R}: Despreciando al vaso, $T_f=283$\,K.

	\item {\bf{Trabajo y calor}}

		Sea $1$\,mol de un gas ideal monoatómico a $p=1$\,atm ocupando un volumen de
		$25$\,L. La energía interna total del gas es $U_A=3800$\,J. El gas es
		calentado mediante una transformación que lo deja en su
		estado final $B$ a una presión de $1$\,atm, un volumen de $75$\,L y la
		energía interna total es de $U_B=11400$\,J. Calcule el trabajo realizado y el
		calor total absorbido por el gas.
		\\{\bf R}: $W=+5066$\,J; $Q=12666$\,J.
\end{enumerate}
\end{document}
%%%%
