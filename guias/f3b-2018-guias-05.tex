\documentclass[a4paper,12pt]{article}
\usepackage[spanish]{babel}
\hyphenation{co-rres-pon-dien-te}
\usepackage[utf8]{inputenc}
\usepackage[T1]{fontenc}
\usepackage{graphicx}
\usepackage[pdftex,colorlinks=true, pdfstartview=FitH, linkcolor=blue,
citecolor=blue, urlcolor=blue, pdfpagemode=UseOutlines, pdfauthor={H. Asorey},
pdftitle={F3B+F4B - Guía 03}]{hyperref}
\usepackage[adobe-utopia]{mathdesign}

\hoffset -1.23cm
\textwidth 16.5cm
\voffset -3.5cm
\textheight 26.0cm

%----------------------------------------------------------------
\begin{document}
\title{
{\normalsize{Universidad Nacional de Río Negro - Profesorado de Física}}\\
Física 3B+4A  2018 \\ Guía 05: Segundo principio
}
\author{Asorey}
\date{30 de Mayo de 2018} 
\maketitle

\begin{enumerate}
	\setcounter{enumi}{38}      %% esta guia va del problema 39 al XX
    \item {\bf{El cuadrado inverso}}
		Una máquina frigorífica está equipada con $n=1000$\,moles de un gas
		ideal di-atómico inicialmente ocupando un volumen $V_A=112$\,m$^3$ a una
		temperatura $T_A=6825$\,K, y que opera con el siguiente ciclo: 1)
		enfriamiento isocórico hasta llegar a un quinto de la temperatura
		inicial; 2) compresión isobárica hasta alcanzar un quinto del volumen
		inicial; 3) calentamiento isocórico; 4) expansión isobárica.
        \begin{enumerate}
            \item Complete el cuadro de estados, encontrando los valores de
                $P$, $V$, $T$ y $n$ para cada uno de los estados $A$, $B$, $C$
                y $D$.
            \item En el diagrama $PV$ ubique los estados y dibuje las
				transformaciones experimentadas por el gas. Luego hágalo para
				un diagrama $TS$.
            \item Complete el cuadro de transformaciones, encontrando los
                cambios de energía interna, calor, trabajo y entropía en cada
                transformación.
			\item Calcule el rendimiento de la máquina frigorífica usando la
				definición de una máquina térmica, $\eta=W/Q_{+}$.
			\item Compare los valores obtenidos con los correspondientes del
				ejercicio 27 (Guía 03).
			\item Calcule el cambio de entropía total del Universo. Explique.
        \end{enumerate}
        % {\bf{R}}: a) $P_A=101325$\,Pa, $V_A=22,4$\,m$^3$, $n_A=1000$\,mol,
        % $T_A=273$\,K; $P_B=506634,4$\,Pa, $V_B=22,4$\,m$^3$, $n_B=1000$\,mol,
        % $T_B=1365$\,K; $P_C=506634,4$\,Pa, $V_C=112$\,m$^3$, $n_C=1000$\,mol,
        % $T_C=6825$\,K; $P_D=101325$\,Pa, $V_D=112$\,m$^3$, $n_D=1000$\,mol,
        % $T_D=1365$\,K; c) $Q_1=22,7$\,MJ, $\Delta U_1=22,7$\,MJ, $W_1=0$;
        % $Q_2=158,9$\,MJ, $\Delta U_2=113,5$\,MJ, $W_2=45,4$\,MJ;
        % $Q_3=-113,5$\,MJ, $\Delta U_3=-113,5$\,MJ, $W_3=0$; $Q_4=-31,8$\,MJ,
        % $\Delta U_4=-22,7$\,MJ, $W_4=-9,1$\,MJ; d) $\eta=36,3/181,6 = 0,2 =
        % 20\%$; $\eta_{\mathrm{Carnot}} = 0,96 = 96\%$;

	\item {\bf{Entropía en aumento, 1}}

		Para mantener la temperatura dentro de una casa a $293$\,K se necesita
		mantener funcionando un sistema de calefacción con una potencia de
		$30$\,kW por día cuando la temperatura en el exterior es de $266$\,K.
		¿Cuál es la tasa de incremento de la entropía total del Universo
		provocada por esta casa?

	\item {\bf{Variación de entropía, I}}

		¿Cuál es la variación de entropía de $1$\,mol de un gas ideal
		monoatómico si su temperatura aumenta de $100$\,K a $300$\,K en una
		transformación: a) isocórica; b) isobárica; c) adiabática y reversible.

	\item {\bf{Variación de entropía, II}}
		
		Calcule la variación de entropía cuando un mol de un gas ideal
		monoatómico se lleva desde $T_A=273$\,K y $P_A=2$\,atm hasta un estado
		$T_B=233$\,K y $P_B=0.4$\,atm.

	\item {\bf{Ciclo reversible}}

		Una determinada cantidad de Helio está inicialmente a una presión
		$P_A=16$\,atm, $T_A=600$\,K y ocupa un volumen $V_A=1$\,L. Se somete a
		una expansión isotérmica de manera cuasiestática (reversible) hasta un
		volumen $V_B=4.0$\,L. Luego, es comprimido cuasi-estática e
		isobáricamente hasta que su volumen $V_C$ y temperatura $T_C$ le
		permiten volver de manera adiabática y reversible al estado original
		$A$. 
		\begin{enumerate}
			\item Complete el cuadro de estados y de transformaciones. Incluya
				para este último los cambios de entropía en cada
				transformación. 
			\item Realice un esquema del ciclo en un diagrama $PV$ y en un
				diagrama $TS$. 
			\item Calcule la eficiencia del ciclo; 
			\item Verifique el cumplimiento de la desigualdad de Clausius en
				este ciclo.
			\item Calcule el cambio de entropía del Universo
		\end{enumerate}

	\item {\bf{Entropía del hielo}}

		Calcule el cambio de entropía de un bloque de hielo de $54.0$\,g que se
		encuentra a $250$\,K cuando es convertido de manera reversible y a
		presión constante, en vapor a $390$\,K.

	\item {\bf{Ciclo de Stirling}}

		Cien moles de un gas ideal biatómico es sometido a un ciclo de Stirling
		completamente reversible. El gas, inicialmente a una presión
		$p_A=150$\,kPa y $T_A=300$\,K, experimenta una compresión isoterma
		hasta alcanzar el estado $B$, seguido de una compresión isócora hasta
		una presión $p_C=3$\,MPa y $T_C=2000$\,K. Luego se realiza una
		expansión isotérmica hasta alcanzar el estado $D$ tal que $V_D=V_A$. El
		ciclo se cierra con una decompresión isócora hasta volver al estado
		$A$.
		\begin{enumerate}
			\item Grafique el ciclo en un diagrama $PV$ y en un diagrama $TS$.
			\item Complete el cuadro de estados y el cuadro de
				transformaciones, calculando además los cambios de entropía del
				gas en cada transformación.
			\item Calcule la variación neta de entropía del Universo
			\item Calcule el rendimiento del ciclo y compárelo con el de un
				ciclo de Carnot equivalente. Analice el resultado en función
				del teorema de Carnot.
		\end{enumerate}

	\item {\bf{Ciclo Otto}}

		Dibuje el ciclo Otto del problema 36 (guía 04) en un diagrama $TS$
		identificando cada transformación y los cambios de energía con el
		medio.

	\item {\bf{Verificando el segundo principio}}

		Compruebe el cumplimiento del segundo principio de la termodinámica en
		para los problemas 26, 27, 28, 29, 30 y 36, verificando que la entropía total
		del universo no decrece en ninguno de esos casos. Identifique los
		ciclos reversibles y los ciclos irreversibles.
	
	\item {\bf{Entropía terrestre}}

		La temperatura promedio de la superficie solar es $5700$\,K y la
		temperatura promedio de la superficie terrestre es de $290$\,K. Si la
		constante solar es de $1.4$\,kW m$^{-2}$, a) estime la cantidad de
		energía que por segundo llega a la Tierra; b) estime la tasa neta de
		incremento de entropía de la Tierra debida a la radiación Solar.

	\item {\bf{Aumento de entropía, II}}
		
		Un gas ideal biatómico se encuentra en el interior de un cilindro de
		paredes adiabáticas. El estado inicial del gas: $P_A=101325$\,Pa,
		$T_A=293$\,K y $V_A=10^{-4}$\,m$^3$. La tapa del cilindro es un pistón
		móvil también adiabático. De manera brusca se coloca una pesa en el
		pistón y la presión en el interior aumenta a $P_B=2 P_A$. Determine el
		estado final del gas, y la variación de entropía del sistema, del medio
		y del Universo, y diga si el proceso fue reversible o irreversible.
		Justifique. Luego, compare los resultados obtenidos con los que se
		obtienen durante una compresión adiabática reversible desde $P_A$ a
		$P_B$.

\end{enumerate}
\end{document}
%%%%
