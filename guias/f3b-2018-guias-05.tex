\documentclass[a4paper,12pt]{article}
\usepackage[spanish]{babel}
\hyphenation{co-rres-pon-dien-te}
\usepackage[utf8]{inputenc}
\usepackage[T1]{fontenc}
\usepackage{graphicx}
\usepackage[pdftex,colorlinks=true, pdfstartview=FitH, linkcolor=blue,
citecolor=blue, urlcolor=blue, pdfpagemode=UseOutlines, pdfauthor={H. Asorey},
pdftitle={F3B+F4B - Guía 03}]{hyperref}
\usepackage[adobe-utopia]{mathdesign}

\hoffset -1.23cm
\textwidth 16.5cm
\voffset -3.5cm
\textheight 26.0cm

%----------------------------------------------------------------
\begin{document}
\title{
{\normalsize{Universidad Nacional de Río Negro - Profesorado de Física}}\\
Física 3B+4A  2018 \\ Guía 05: Segundo principio
}
\author{Asorey}
\date{23 de Mayo de 2018} 
\maketitle

\begin{enumerate}
	\setcounter{enumi}{38}      %% esta guia va del problema 39 al XX
    \item {\bf{El cuadrado inverso}}
		Una máquina frigorífica está equipada con $n=1000$\,moles de un gas
		ideal di-atómico incialmente ocupando un volumen $V_A=112$\,m$^3$ a una
		temperatura $T_A=6825$\,K, y que opera con el siguiente ciclo: 1)
		enfriamiento isocórico hasta llegar a un quinto de la temperatura
		inicial; 2) compresión isobárica hasta alcanzar un quinto del volumen
		inicial; 3) calentamiento isocorico; 4) expansión isobárica.
        \begin{enumerate}
            \item Complete el cuadro de estados, encontrando los valores de
                $P$, $V$, $T$ y $n$ para cada uno de los estados $A$, $B$, $C$
                y $D$.
            \item En el diagrama $P-V$ ubique los estados y dibuje las
                transformaciones experimentadas por el gas.
            \item Complete el cuadro de transformaciones, encontrado los
                cambios de energía interna, calor y trabajo en cada
                transformación.
			\item Calcule el rendimiento de la máquina frigorífica usando la
				definición de una máquina térmica, $\eta=W/Q_{+}$.
			\item Compare los valores obtenidos con los correspondientes del
				ejercicio 27 (Guía 03).
        \end{enumerate}
        % {\bf{R}}: a) $P_A=101325$\,Pa, $V_A=22,4$\,m$^3$, $n_A=1000$\,mol,
        % $T_A=273$\,K; $P_B=506634,4$\,Pa, $V_B=22,4$\,m$^3$, $n_B=1000$\,mol,
        % $T_B=1365$\,K; $P_C=506634,4$\,Pa, $V_C=112$\,m$^3$, $n_C=1000$\,mol,
        % $T_C=6825$\,K; $P_D=101325$\,Pa, $V_D=112$\,m$^3$, $n_D=1000$\,mol,
        % $T_D=1365$\,K; c) $Q_1=22,7$\,MJ, $\Delta U_1=22,7$\,MJ, $W_1=0$;
        % $Q_2=158,9$\,MJ, $\Delta U_2=113,5$\,MJ, $W_2=45,4$\,MJ;
        % $Q_3=-113,5$\,MJ, $\Delta U_3=-113,5$\,MJ, $W_3=0$; $Q_4=-31,8$\,MJ,
        % $\Delta U_4=-22,7$\,MJ, $W_4=-9,1$\,MJ; d) $\eta=36,3/181,6 = 0,2 =
        % 20\%$; $\eta_{\mathrm{Carnot}} = 0,96 = 96\%$;

	\item 
\end{enumerate}
\end{document}
%%%%
